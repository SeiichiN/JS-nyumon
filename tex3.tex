\documentclass[dvipdfmx]{jsarticle}

\title{JavaScript入門その1}
\author{原著者:清水健二 / 改訂:糠山誠一}
\date{2020-06-13}
\usepackage{tcolorbox}
\usepackage{color}
\usepackage{listings, plistings}

% Java
\lstset{% 
  frame=single,
  backgroundcolor={\color[gray]{.9}},
  stringstyle={\ttfamily \color[rgb]{0,0,1}},
  commentstyle={\itshape \color[cmyk]{1,0,1,0}},
  identifierstyle={\ttfamily}, 
  keywordstyle={\ttfamily \color[cmyk]{0,1,0,0}},
  basicstyle={\ttfamily},
  breaklines=true,
  xleftmargin=0zw,
  xrightmargin=0zw,
  framerule=.2pt,
  columns=[l]{fullflexible},
  numbers=left,
  stepnumber=1,
  numberstyle={\scriptsize},
  numbersep=1em,
  language={Java},
  lineskip=-0.5zw,
  morecomment={[s][{\color[cmyk]{1,0,0,0}}]{/**}{*/}},
}
%\usepackage[dvipdfmx]{graphicx}
\usepackage{url}
\usepackage[dvipdfmx]{hyperref}
\usepackage{amsmath, amssymb}
\usepackage{itembkbx}
\usepackage{eclbkbox}	% required for `\breakbox' (yatex added)
\fboxrule=0.5pt
\parindent=1em
\begin{document}

%% 修正時刻: Sat Jun 13 16:00:25 2020


\section{DOMを操作する}

ブラウザの画面に表示するために \verb!document.write(`<li>${listText}</li>`)! などと
していましたが、document.write は その挙動の不安定さからHTML5 では \textbf{非推奨}
となっています。

では、ブラウザの画面上に文字を出力したり、\verb!<h1>! などのタグを出力したりするには
どうするかというと、タグやid、class で構成された要素を取得し、その取得した要素に対して
さまざまな文字を追加したり、\verb!<li>!などのタグを追加したりします。

ブラウザは文章や画像などのコンテンツとHTMLの構造からなるオブジェクトで、その表現形態を
\textbf{DOM} (Document Object Model) といいます。

\textbf{DOM} は、HTMLのための''プログラミングインターフェース''と言えます。
言い換えるなら、ウェブページを''表現、保存、操作する方法''と言えます。
\footnote[1]{https://developer.mozilla.org/ja/docs/Web/API/Document\_Object\_Model/Introduction}





\end{document}

%% 修正時刻: Sat May  2 15:10:04 2020


%% 修正時刻: Sat Jun 13 20:56:45 2020
