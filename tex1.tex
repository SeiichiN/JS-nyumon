\documentclass[dvipdfmx]{jsarticle}

\title{JavaScript入門その1}
\author{原著者:清水健二 / 改訂:糠山誠一}
\date{2020-06-13}
\usepackage{tcolorbox}
\usepackage{color}
\usepackage{listings, plistings}

% Java
\lstset{% 
  frame=single,
  backgroundcolor={\color[gray]{.9}},
  stringstyle={\ttfamily \color[rgb]{0,0,1}},
  commentstyle={\itshape \color[cmyk]{1,0,1,0}},
  identifierstyle={\ttfamily}, 
  keywordstyle={\ttfamily \color[cmyk]{0,1,0,0}},
  basicstyle={\ttfamily},
  breaklines=true,
  xleftmargin=0zw,
  xrightmargin=0zw,
  framerule=.2pt,
  columns=[l]{fullflexible},
  numbers=left,
  stepnumber=1,
  numberstyle={\scriptsize},
  numbersep=1em,
  language={Java},
  lineskip=-0.5zw,
  morecomment={[s][{\color[cmyk]{1,0,0,0}}]{/**}{*/}},
}
%\usepackage[dvipdfmx]{graphicx}
\usepackage{url}
\usepackage[dvipdfmx]{hyperref}
\usepackage{amsmath, amssymb}
\usepackage{itembkbx}
\usepackage{eclbkbox}	% required for `\breakbox' (yatex added)
\fboxrule=0.5pt
\parindent=1em
\begin{document}

%% 修正時刻: Sat Jun 13 16:00:25 2020


\section{JavaScriptとは}

JavaScriptは、クライアントPCのインターネットブラウザ上で動作するスクリプト言語の一種です。
(「Java」と付いてはいますが、Javaとは全く別の言語です) \\
JavaScriptで記述された処理は、ブラウザ上でユーザーからの動作を\textgt{イベント} として
受け取った時に実行され、処理結果は表示されている\textgt{Webページに直接反映}されます。
この動作はクライアント内で完結しており、\textgt{Webページを再読み込みすることなく処理が
行われます。}

\section{JavaScriptの変遷}

\begin{table}[h]
  \begin{tabular}{|p{40mm}|p{40mm}|p{70mm}|}
   \hline
   \multicolumn{1}{|c|}{年代} & \multicolumn{1}{|c|}{時代} & \multicolumn{1}{|c|}{概要} \\ \hline
   1990年代後半 & 初期JS全盛期 &
           ブラウザNetscapeに搭載。HTMLに華やかな効果を搭載したり、簡単なフォームチェックが行えるようになった。 \\ \hline
   2000年代前半 & Flath、IE全盛期 &
           ブラウザInternet Explorerが独自仕様のJSを搭載。Netscapeとの互換性なくなる。Flashの台頭によりJavaScriptは衰退。 \\ \hline
   2005年 & Web2.0時代 &
           Ajax技術の登場。GoogleマップやYouTubeなど。再びJavaScriptが注目される。 \\ \hline
   2000年代後半 & jQueryの登場 HTML5 &
           JavaScriptのライブラリとしてjQueryが登場。容易に動的なWebページが作れるようになる。また、HTML5が普及しだし、JavaScriptの活躍の場が広がる。 \\ \hline
   2010年前半 & スマホの普及 &
           iPhone、Android等のスマートフォンが登場。セキュリティ性に問題があったFlashへの対応を切り捨てたことで、Flashは衰退。代わりに HTML5 + JavaScript による動的Webページが作成されるようになる。 \\ \hline
   2015年 & ECMAScript2015の制定 &
           新しい時代のWeb開発に対応するため、JavaScriptの機能や文法を大幅に拡充。通称ES2015 または ES6。 \\ \hline
   2010年代後半 & JSの多機能化 &
           node.js の登場により、JavaScriptでサーバサイドのプログラムが組めるようになる。また、アプリケーションを作る枠組(フレームワーク)として、React.js、AngularJSなどが登場。 \\ \hline
  \end{tabular}
\end{table}

JavaScriptとは、Webアプリケーションを作成するための根幹技術。

まず、「とりあえず動かせる」「なんとなく書ける」というレベルを目指すこととします。


\section{JavaScriptを書くための準備}

\subsection{HTMLファイルの準備}

以下の内容で start.html を作成してください。

\begin{lstlisting}
 <!doctype html>
 <html lang="ja">
 <head>
   <meta charset="utf-8">
   <title>JavaScript</title>
   <link rel="stylesheet" href="style.css">
   <script>

   </script>
 </head>
 <body>
   <h1 id="heading1">見出し</h1>
   <button onclick="kansu1()">ボタン</button>
   <ul>
     <li>HTMLで書きました</li>
     <!-- JSはbody内にも書ける -->
     <script

     </script>
   </ul>
   <img id="img1" src="images/4.jpg" alt="">
   <!-- bodyの一番下に書く場合も多い -->
   <script>

   </script>
 </body>
 </html>
\end{lstlisting}

\subsection{CSSファイルの作成}

''style.css''を作成します。お好きなようにデザインしてください。

 \begin{lstlisting}
  /* 画像は大きいので、300pxくらいで。 */
  img {
    width: 300px;
  }

  /* リンク以外のところでも、指の形になります */
  h1, img {
    cursor: pointer;
  }
  
 \end{lstlisting}

\include{end}

%% 修正時刻: Sat Jun 13 07:56:46 2020
